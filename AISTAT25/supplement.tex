\documentclass[twoside]{article}

\usepackage{aistats2024}
\usepackage{enumitem}
\usepackage{multicol}
\usepackage{times}
\usepackage{amssymb}
\usepackage{graphicx}

\def\R{{\mathbb{R}}}
\def\pr{{\rm Pr}}
\def\E{{\mathbb E}}
\def\X{{\mathcal X}}
\def\Y{{\mathcal Y}}
\def\H{{\mathcal H}}
\def\G{{\mathcal G}}
\def\B{{\mathcal B}}
\def\yh{{\widehat{y}}}
\def\bias{{\rm bias}}
\def\supp{{\rm supp}}
\def\dist{{\rm dist}}
\def\sign{{\rm sign}}
\def\vol{{\rm vol}}
\def\PL{{\mbox{\rm PL}}}

\newtheorem{thm}{Theorem}
\newtheorem{lemma}[thm]{Lemma}
\newtheorem{cor}[thm]{Corollary}
%\newtheorem{claim}[thm]{Claim}
\newtheorem{defn}[thm]{Definition}
\newtheorem{assump}{Assumption}
\newenvironment{proof}{\noindent {\sc Proof:}}{$\Box$ \medskip}

\newcommand{\comment}[2]{ {\bf #1 :} #2}
\newcommand{\yoav}[1]{\comment{Yoav}{\em #1}}
\newcommand{\sanjoy}[1]{\comment{blue}{Sanjoy}{#1}}

\DeclareMathOperator*{\argmax}{arg\,max}

% If your paper is accepted, change the options for the package
% aistats2022 as follows:
%
%\usepackage[accepted]{aistats2024}
%
% This option will print headings for the title of your paper and
% headings for the authors names, plus a copyright note at the end of
% the first column of the first page.

% If you set papersize explicitly, activate the following three lines:
%\special{papersize = 8.5in, 11in}
%\setlength{\pdfpageheight}{11in}
%\setlength{\pdfpagewidth}{8.5in}

% If you use natbib package, activate the following three lines:
%\usepackage[round]{natbib}
%\renewcommand{\bibname}{References}
%\renewcommand{\bibsection}{\subsubsection*{\bibname}}

% If you use BibTeX in apalike style, activate the following line:
%\bibliographystyle{apalike}

\begin{document}

% If your paper is accepted and the title of your paper is very long,
% the style will print as headings an error message. Use the following
% command to supply a shorter title of your paper so that it can be
% used as headings.
%
%\runningtitle{I use this title instead because the last one was very long}

% If your paper is accepted and the number of authors is large, the
% style will print as headings an error message. Use the following
% command to supply a shorter version of the authors names so that
% they can be used as headings (for example, use only the surnames)
%
%\runningauthor{Surname 1, Surname 2, Surname 3, ...., Surname n}

% Supplementary material: To improve readability, you must use a single-column format for the supplementary material.
\onecolumn
\aistatstitle{Non-parametric active learning without smoothness\\
Supplementary Materials}

\section{Detailed algorithm}

We now describe the detailed active learning algorithm as shown in
Figs~\ref{alg:main}, \ref{alg:sampling} and~\ref{alg:bias-estimate}.
The algorithm makes two types of query. \emph{Background queries} are random draws
from $X$ and correspond to passive learning. \emph{Focused queries}
are made in the vicinity of ``uncertain'' points and correspond to
active learning.

Many of the elements of the detailed algorithm have been introduced in the simplified version.
Specifically $U_\ell,S,\PL_\ell(x),\yh_\ell(x)$, are as defined there.

However, in the detailed algorithm all of the levels are considered at each iteration.
On each iteration of the main loop (at most) one focused
query is made as well as a background query. The focused query comes
from the lowest-numbered uncertainty region $U_\ell$ that is nonempty.

Once these queries are made, bias estimates $\widehat{\eta}(B)$ are
updated using the Poisson method,\\
$\yh(B), \PL_\ell(x), \yh_\ell(x), U_\ell$ are updated
accordingly  and the next iteration begins.

The querying process can be stopped at any time, whereupon labels are assigned as follows:
\begin{equation}
\yh(x) = 
\left\{
\begin{array}{cl}
\yh_\ell(x) & \mbox{largest $\ell$ with $\yh_\ell(x) \in \{-1,+1\}$, if such an $\ell$ exists} \\
0 & \mbox{``don't know'', otherwise}
\end{array}
\right.
\label{eq:final-label}
\end{equation}


%%%%%%%%%%%%%%%%%%%%%%%%%%%%%%%%%%%%

\begin{figure}[t]
  \begin{framebox}
%    \begin{minipage}
% [t]{6in}
%\begin{center}
%\begin{minipage}[t]{6in}
%\vspace{.05in}

Initialize uncertainty regions at all levels: $U_0 = X$ and $U_\ell = \emptyset$ for $\ell \geq 1$

%\vspace{.1in}

Initialize labels at all levels to ``unavailable'': \ $\yh_\ell(x) = \bot$ for all $x$ and $\ell \geq 0$

%\vspace{.1in}

Repeat:
\begin{itemize}[leftmargin=0.5cm]
\item If there is a level $\ell \geq 0$ such that $U_\ell \neq \emptyset$:
\begin{itemize}[leftmargin=0.25cm]
\item For the smallest such level $\ell'$, run {\tt Focused-query}($\ell', U_{\ell'}$) // Fig.~\ref{alg:sampling}
\end{itemize}
\item Run {\tt Background-query} // Fig.~\ref{alg:sampling}
\item Update labels:
\begin{itemize}[leftmargin=0.25cm]
\item Update bias-estimates $\yh(B)$ // Fig.~\ref{alg:bias-estimate}
\item For each $x \in X$ and level $\ell \geq 0$ for which all $\{\yh(B): B \in \B_\ell(x)\}$ are available:
\begin{align}
\PL_\ell(x) &= \{s \in \{-1, +1\}: \ \mbox{there exists $B \in \B_{\leq \ell}(x)$ s.t. $\yh(B) = s$ and} \notag \\ 
             &                       \hspace{1.5in}\mbox{no $B' \in \B_{\leq \ell}(x)$ has $X_{B'} \subsetneq X_B$.} \} \label{eq:PL} \\ 
\yh_\ell(x) &= 
\left\{
\begin{array}{cl}
+1 & \mbox{if $\PL_\ell(x) = \{+1\}$} \\
-1 &  \mbox{if $\PL_\ell(x) = \{-1\}$} \\
0 & \mbox{if $\PL_\ell(x) = \{\}$} \\
! & \mbox{if $\PL_\ell(x) = \{-1,+1\}$}
\end{array}
\right.
\label{eq:provisional-label}
\end{align}

\end{itemize}
\item Update uncertainty regions:
\begin{itemize}[leftmargin=0.25cm]
\item $U_0 = \{x \in X: \yh_0(x) = \bot\}$
\item For all levels $\ell \geq 1$: \ $U_\ell = \{x \in X: \yh_{\ell-1}(x) = \,!, \ \yh_\ell(x) = \bot\}$
\end{itemize}
\end{itemize}

%\end{minipage}
%\end{center}
%\end{minipage}
\end{framebox}
\caption{The active learning algorithm. Each iteration of the main loop makes (at most) one focused query and one background query.}
\label{alg:main}
\end{figure}

%%%%%%%%%%%%%%%%%%%%%%%


\subsubsection{Poisson Sampling}
\label{sec:poisson}


The uncertainty region at level $\ell$ consists of the known-unknowns from level $\ell-1$, that is, points $x \in X$ with $\yh_{\ell-1}(x) = \, !$. This region is revealed to us piecemeal, a few points at a time. To get good label complexity we need to actively query the part of it we know.

We manage this through \emph{Poisson sampling} (Fig.~\ref{alg:sampling}). To infer the bias of a ball $B$, we need labels for $k = \tilde{O}(1/\gamma^2)$ random points in it. Balls at level $\ell$ have roughly $n/2^{\ell+1}$ points, so we need about $2^{\ell+1} k/n$ fraction of the ball to be queried. Instead, we query every point in the ball with probability $\tau_\ell = \min(2^{\ell+2}k/n, 1)$, \emph{independently}. Specifically, we query points $x$ in the ball with $T_x \leq \tau_\ell$, where each $x \in X$ is assigned a uniform-random value $T_x \in [0,1]$ at the outset of the algorithm. 

Once a point is queried, it is never queried again.

\begin{figure}[t]
\framebox{
\begin{minipage}[t]{6in}

\vspace{.05in}
\emph{Initialization:}
\begin{itemize}[leftmargin=0.5cm]
\item Set $Q = \emptyset$ (points queried so far)
\item For each $x \in X$: choose $T_x \sim \mbox{uniform}([0,1])$
\end{itemize}

\vspace{.05in}
{\bf Background-query}

\begin{itemize}[leftmargin=0.5cm]
\item Query the next unlabeled point in $X \setminus Q$, ordered by increasing $T_x$ values, and add to $Q$
\end{itemize}

\vspace{.05in}
{\bf Focused-query}($\ell, U$)

\begin{itemize}[leftmargin=0.5cm]
\item Define querying region:
$$ S =  \bigcup_{x \in U} \bigcup_{B \in \B_\ell(x)} \{z \in X_B: T_z \leq \tau_\ell\} $$
\item Query the next unlabeled point in $S \setminus Q$, ordered by increasing $T_x$ values, and add to $Q$
\end{itemize}

\end{minipage}}
\caption{The two sampling procedures. Each $x \in X$ has an associated r.v. $T_x$ drawn uniformly from $[0,1]$. The smaller $T_x$, the earlier $x$ is likely to be queried. \emph{Background queries} are drawn at random from $X$. \emph{Focused queries} are from the uncertainty region at a given level $\ell$, and use level-based thresholds $\tau_\ell = \min(2^{\ell+2}k/n, 1)$, where $k = \tilde{O}(1/\gamma^2)$.} 
\label{alg:sampling}
\end{figure}


\begin{figure}
\framebox{
\begin{minipage}[t]{6in}
\vspace{.05in}
\begin{itemize}[leftmargin=0.25cm]
\item Initially $\yh(B) = \bot$
\item When all of $\{z \in X_B: T_z \leq \tau_\ell\}$ is queried, let $\widehat{\eta}(B)$ be the mean of these labels and set
$$ \yh(B)
= 
\left\{
\begin{array}{ll}
\sign(\widehat{\eta}(B)) & \mbox{if $|\widehat{\eta}(B)| \geq \gamma/2$} \\
0 & \mbox{otherwise}
\end{array}
\right.
$$
\end{itemize}
\end{minipage}}
\caption{The qualitative bias $\yh(B)$ of a ball $B \in \B_\ell$; see Fig.~\ref{alg:sampling} for $\tau_\ell$.}
\label{alg:bias-estimate}
\end{figure}

%%%%%%%%%%%%
\section{Analysis Details: Discrete setting}

\subsection{Technicalities: discrete setting}

\subsubsection{Large deviation bounds for the discrete setting}

\begin{lemma}
Fix a confidence parameter $0 < \delta < 1$ and a positive integer $k \geq 6 \ln (4/\delta)$. 

Let $x_1, \ldots, x_m$ be any collection of points. Suppose that the labels $Y_i \in \{-1,1\}$ of these points are independent, with $\E Y_i = \eta(x_i)$. Define
$$ \eta_o = \frac{1}{m} \left( \eta(x_1) + \cdots + \eta(x_m) \right) .$$
Now consider the following estimator $Z$ of this quantity:
\begin{itemize}
\item Each $x_i$ is chosen with probability $q > 0$, independently. Let $N$ be the number of selected points.
\item If $N > 0$, the labels $Y_i$ of the selected points are obtained, and $Z$ is their average.
\end{itemize}
If $qm \geq k + \sqrt{6k \ln (4/\delta)}$, with probability at least $1-\delta$, 
\begin{enumerate}
\item[(a)] $N \geq k$, and 
\item[(b)] $| Z - \eta_o| < \sqrt{(48/k) \ln (4/\delta)}$.
\end{enumerate}
\label{lemma:large-deviation-discrete}
\end{lemma}
\begin{proof}
Let's start with (a). Define $c = \sqrt{3 \ln (4/\delta)}$. We'll take $qm = k + \sqrt{6k \ln (4/\delta)} = k + c \sqrt{2k}$ since this is the worst case. By assumption, $k \geq 2c^2$ and thus $qm \leq 2k$.

Now, $N$ has a binomial($m,q$) distribution. By a multiplicative Chernoff bound, for $0 < \epsilon < 1$, we have
\begin{align*}
\pr(N \geq qm(1+\epsilon)) &\leq e^{-qm\epsilon^2/3} \\
\pr(N \leq qm(1-\epsilon)) &\leq e^{-qm\epsilon^2/2}
\end{align*}
Take $\epsilon = c/\sqrt{qm}$; by the lower bound on $k$, we have $\epsilon \leq 1/2$. Recalling the choice of $c$, we see that with probability at least $1-\delta/2$, we get 
$$(1-\epsilon) qm < N < (1+\epsilon) qm .$$
The lower bound implies $N > qm(1-\epsilon) = qm - c\sqrt{qm} = k + c \sqrt{2k} - c \sqrt{qm} \geq k$.

For (b), define $C_1, \ldots, C_m \in \{0,1\}$ as random variables indicating whether the corresponding points were selected; that is, $C_i = {\bf 1}(\mbox{$x_i$ was selected})$. The sum of the obtained labels is then $S = C_1 Y_1 + \cdots + C_m Y_m$. Notice that these $C_iY_i \in \{-1,0,1\}$ are independent with $\E[C_iY_i] = q \eta(x_i)$ and $\E[(C_iY_i)^2] = \E[C_i] = q$. Thus their sum $S$ has expectation
$$ \E [S] = \sum_{i=1}^m \E[C_i Y_i] = \sum_{i=1}^m q \eta(x_i) = qm \eta_o $$
and variance
$$ \mbox{var}(S) = \sum_{i=1}^m \mbox{var}(C_iY_i) \leq qm .$$
We can bound the concentration of $S$ around its expected value using Bernstein's inequality, by which
$$ \pr(|S - \E[S]| \geq t) \leq 2 \exp \left( - \frac{t^2}{2(\mbox{var}(S) + t/3)} \right) .$$
Using $t = \epsilon qm$, we then have that $|S - qm \eta_o| \leq \epsilon qm$ with probability at least $1-\delta/2$.

Combining with the high-probability bound on $N$ above, we get
$$ \frac{qm \eta_o - \epsilon qm}{qm(1+\epsilon)} < \frac{S}{N} < \frac{qm \eta_o + \epsilon qm}{qm(1-\epsilon)},$$
whereupon (recalling $Z = S/N$)
$$ 
\eta_o \left( \frac{1}{1+\epsilon} -1 \right) - \frac{\epsilon}{1+\epsilon} < Z - \eta_o < \eta_o \left( \frac{1}{1-\epsilon} - 1 \right) + \frac{\epsilon}{1-\epsilon} .$$
Since $|\eta_o| \leq 1$,
$$ |Z - \eta_o | < \max \left( \frac{2\epsilon}{1+\epsilon}, \frac{2\epsilon}{1-\epsilon} \right)
\leq 
4 \epsilon,$$
as claimed.  
\end{proof}


\subsubsection{Accuracy of bias estimates: Proof of Lemma~\ref{lemma:accurate-bias-estimates}}

We will use Lemma~\ref{lemma:large-deviation-discrete} to obtain a uniform guarantee on the bias estimates for all balls $B \in \B$.

Recall that each point $x \in X$ gets a label $Y \in \{-1,+1\}$ according to the distribution
$$ \eta(x) = \E[y | x].$$ 
For any $B \in \B$ with $X_B \neq \emptyset$, let $\eta_X(B)$ be the average $\eta$-value of the points in $B$, that is,
$$ \eta_X(B) = \frac{1}{|X_B|} \sum_{x \in X_B} \eta(x) .$$

For any $B \in \B_\ell$, define its \emph{query-set} to be $\Gamma(B) = \{z \in X_B: T_z \leq \tau_\ell\}$, where the sampling threshold $\tau_\ell$ for level $\ell$ is defined as:
\begin{equation}
\tau_\ell = \min \left( \frac{2^{\ell+2}k}{n}, \ 1 \right)
\end{equation}
for some $k$. We base our bias-estimate for $B$ on the labels of points in $\Gamma(B)$.

For what follows, define
\begin{equation*}
c = \sqrt{3 \ln \frac{4|\B|}{\delta}} .
\end{equation*}

\begin{lemma}
Suppose $k \geq 2c^2$. With probability at least $1-\delta$, the following is true for all $B \in \B$ with $|X_B| \geq k$:
\begin{enumerate}
\item[(a)] The query set $\Gamma(B)$ has size at least $k$.
\item[(b)] The average label on $\Gamma(B)$, call it $\widehat{\eta}(B)$, satisfies
$$ \left| \widehat{\eta}(B) - \eta_X(B) \right| \leq \frac{4c}{\sqrt{k}}.$$
\end{enumerate}
\label{lemma:large-deviation-bounds}
\end{lemma}
\begin{proof}
Pick $B \in \B$. There are two cases to consider.

Case 1: $|X_B| \geq 2k$. The choice of $\tau_\ell$ then ensures $|X_B| \tau_\ell \geq 2k$, so that $|\widehat{\eta}(B) - \eta_X(B)|$ can be bounded by applying Lemma~\ref{lemma:large-deviation-discrete} to the points $X_B$ with sampling probability $q = \tau_\ell$.

Case 2: $k \leq |X_B| < 2k$. In this case, $B$ lies at a level $\ell$ for which $\tau_\ell = 1$. Lemma~\ref{lemma:large-deviation-discrete} does not apply directly, but its conclusion still holds. In particular, the query-set $\Gamma(B)$ is all of $X_B$, and the same Bernstein bound from the proof of Lemma~\ref{lemma:large-deviation-discrete} can be again be applied.

To complete the proof, we take a union bound over all $B \in \B$.
\end{proof}

The following corollary of Lemma~\ref{lemma:large-deviation-bounds} is immediate.
\begin{cor}
Suppose that $k \geq (8c/\gamma)^2$. For each $B \in \B$, let $\widehat{\eta}(B)$ be the average label on the query-set $\Gamma(B)$, and define the bias-estimate $\yh(B)$ as follows:
$$ \yh(B)
= 
\left\{
\begin{array}{ll}
\sign(\widehat{\eta}(B)) & \mbox{if $|\widehat{\eta}(B)| \geq \gamma/2$} \\
0 & \mbox{otherwise}
\end{array}
\right.
$$
With probability $\geq 1-\delta$, all bias estimates $\yh(B)$, for $B \in \B$, are $\gamma$-accurate.
\label{cor:accurate-bias-estimates}
\end{cor}
\begin{proof}
By the choice of $k$, we have from Lemma~\ref{lemma:large-deviation-bounds} that 
$| \widehat{\eta}(B) - \eta_X(B) | < \gamma/2$
for all $B \in \B$.
\end{proof}

\subsection{Critical levels: Proof of Lemma~\ref{lemma:boundary}}

For part (a), note that some $B_o \in \B_{L_1(x)}(x)$ has significant bias (that is, bias $\geq \gamma$) towards the correct label $s(x)$, as does any ball $B \in \B(x)$ contained within it. For $\ell \geq L_1(x)$, the set $\{B \in \B_{\leq \ell}(x): X_{B} \subseteq X_{B_o}\}$ is nonempty (it contains $B_o$); pick any minimal ball $B$ within it. Then $\yh(B) = s(x)$ by the $\gamma$-accuracy of bias estimates (Lemma~\ref{lemma:accurate-bias-estimates}) and thus $s(x) \in \PL_\ell(x)$.

For (b), take any $\ell \geq L_2(x)$. Consider any $B \in \B_{\leq \ell}(x)$ for which $s(x) \cdot \eta_X(B) < 0$. By definition of $L_2(x)$, this $B$ must lie in $B_{< L_2(x)}(x)$, and moreover there must exist $B' \in \B_{\leq L_2(x)}(x) \subset \B_{\leq \ell}(x)$ with $X_{B'} \subset X_B$ and $s(x) \cdot \eta_X(B') \geq 0$. Thus, any minimal $B \in \B_{\leq \ell}(x)$ has bias $\geq 0$ towards the correct label $s(x)$, whereupon $\yh(B) \in \{0, s(x)\}$ by the $\gamma$-accuracy of bias estimates. Therefore, $-s(x) \not\in \PL_\ell(x)$.

\subsection{The region of focused sampling: Proof of Lemma~\ref{lemma:focused}}

Let $\overline{U}_\ell$ denote the set of all points that are ever (in any round of sampling) in the uncertainty set at level $\ell$. From Lemma~\ref{lemma:boundary}(b), we see that any $x$ with $L_2(x) < \ell$ has $\yh_{\ell-1}(x) \neq \ !$ and thus never makes it into the uncertainty set at level $\ell$. In short,
\begin{equation}
\overline{U}_{\ell} \subset \{x \in X: L_2(x) \geq \ell\}.
\label{eq:uncertainty}
\end{equation}

We see from the {\tt Focused-query} subroutine (Figure~\ref{alg:sampling}) that all focused samples at level $\ell$ lie in
$$
\bigcup_{x \in \overline{U}_\ell} \bigcup_{B \in \B_\ell(x)} \{z \in X_B: T_z \leq \tau_\ell \}
\ 
\subset 
\ 
\bigcup_{x \in X: L_2(x) \geq \ell} \bigcup_{B \in \B_\ell(x)} \{z \in X_B: T_z \leq \tau_\ell \}
\ 
=
\ 
\{z \in \Delta_\ell: T_z \leq \tau_\ell \}.
$$

\subsection{A generic label complexity bound}

We start by showing that various subsets of interest contain roughly the expected number of points at each level.
\begin{lemma}
With probability at least $1-2(\lg (n/k))e^{-k/3}$, the following hold for all levels $0 \leq \ell \leq \lg (n/2k)$.
\begin{enumerate}
\item[(a)] $|\{x \in X: T_x \leq \tau_\ell\}| < 2 n \tau_\ell$.
\item[(b)] If $\Delta_\ell \neq \emptyset$ then $|\{x \in \Delta_{\ell}: T_x \leq \tau_\ell\}| < 2 |\Delta_\ell| \tau_\ell$.
\end{enumerate}
\label{lemma:level-distribution}
\end{lemma}
\begin{proof} 
Pick any subset $S \subset X$ and let $m = |S|$. Then $|\{x \in S: T_x \leq \tau_\ell\}|$ has a $\mbox{binomial}(m, \tau_\ell)$ distribution with expectation $m \tau_\ell$. The probability that it is greater than or equal to twice its expected value is, by a multiplicative Chernoff bound, at most $e^{-m \tau_{\ell}/3}$, which is $\leq e^{-k/3}$ as long as $m \tau_\ell \geq k$.

Both parts follow from this principle; and we take a union bound over all $\lg (n/k)$ levels. For (b), we need to check that $|\Delta_\ell| \tau_\ell \geq k$. To see this, observe from the definition (\ref{eq:sampling-region}) of $\Delta_\ell$ that if it is non-empty, then it contains $X_B$ for at least one ball $B \in \B_\ell$, and every such ball has at least $n/2^{\ell+1}$ points. Combining this with the definition $\tau_\ell = \min(2^{\ell+2}k/n, 1)$ yields $|\Delta_\ell| \tau_\ell \geq k$.
\end{proof}

Theorem~\ref{thm:label-complexity} is a restated version of the following.
\begin{thm}
Suppose the active learning algorithm makes $0 < m \leq n$ queries. Then all points $x \in X$ with $L_1(x) \leq \ell_1$ and $L_2(x) \leq \ell_2$ will get Bayes-optimal labels $\yh(x) = g^*(x)$, where
$$ \ell_1 = \left\lfloor \lg \frac{m}{32k} \right\rfloor $$
and $\ell_2$ is the largest integer $\leq \lg (n/2k)$ such that
$$ \sum_{\ell = \ell_1 + 1}^{\ell_2} |\Delta_{\ell}| \, \tau_{\ell} \ < \ \frac{m}{8} .$$
\label{thm:label-complexity-0}
\end{thm}

\begin{proof}
Denote the first $m/2$ queries by {\it phase one} and the second $m/2$ by {\it phase two}. We will analyze the effect of background sampling in phase one and focused sampling in phase two. We start with the former.

Of the $m/2$ queries in phase one, at least $m/4$ will be background samples. Therefore the $m/4$ points with lowest $T_x$ values are guaranteed to be queried.

Now, for $\ell_1$ as defined, we have that $\tau_{\ell_1} \leq m/8n$ and thus from Lemma~\ref{lemma:level-distribution}(a) that at most $m/4$ points in $X$ satisfy $T_x \leq \tau_{\ell_1}$. Therefore all such points are queried in phase one, and all label-estimates $\{\yh_{\ell_1}(x): x \in X\}$ are set.

It then follows from Lemma~\ref{lemma:boundary}(a) that the following hold for any $x \in X$ with $L_1(x) \leq \ell_1$:
\begin{enumerate}
\item[(a)] By the end of phase one, $\yh_{\ell_1}(x) \in \{g^*(x), !\}$.
\item[(b)] For any $\ell > \ell_1$, when $\yh_\ell(x)$ becomes available, it lies in $\{g^*(x), !\}$.
\item[(c)] If $x$ ever leaves the combined uncertainty region $U = \cup_\ell U_\ell$ during phase two, then its final label as defined in (\ref{eq:final-label}) is henceforth always $\yh(x) = g^*(x)$.
\end{enumerate}

Now let's move on to phase two. Let $A = \{x \in X: L_1(x) \leq \ell_1, L_2(x) \leq \ell_2\}$. We will show that every point in $A$ must leave the uncertainty region $U$ at some time during phase two. From (c), we can conclude that all these points have their final labels set correctly, once and for all.

We will break the argument into two cases. 

{\it Case 1:} Fewer than $m/4$ focused queries are made in phase two. This can only happen if some round of sampling has an empty uncertainty set, meaning that all of $A$ has left $U$ at that point.

{\it Case 2:} A full $m/4$ focused queries are made in phase two. By the analysis of phase one, none of these queries can be at level $\leq \ell_1$ and by Lemma~\ref{lemma:focused}, the total number of possible focused queries at levels $\ell_1+1$ through $\ell_2$ inclusive is at most
$$\sum_{\ell=\ell_1+1}^{\ell_2} |\{z \in \Delta_{\ell}: T_z \leq \tau_\ell \}| 
\ \leq \ 
\sum_{\ell=\ell_1+1}^{\ell_2} 2 |\Delta_{\ell}| \tau_\ell
\ < \ 
\frac{m}{4} ,
$$
where the first inequality is from Lemma~\ref{lemma:level-distribution}(b). Thus at least one query in phase two must be at level $\ell_2 +1$. When this query is made, every $U_\ell$ with $\ell \leq \ell_2$ must be empty and thus all of $A$ must have left the uncertainty region; recall from (\ref{eq:uncertainty}) that no $x \in A$ can be part of $U_\ell$ for $\ell > \ell_2 \geq L_2(x)$. 

Thus every $x \in A$ must leave the uncertainty region at some point in phase two, and their final labels are subsequently set correctly.
\end{proof}


\subsection{One-dimensional monotonic $\eta$: Proof of Theorem~\ref{thm:oned-monotonic}}

We begin by bounding the critical levels $L_1$ and $L_2$ for points in $X$.
\begin{lemma}
Pick any $x \in [0, 1]$.
\begin{enumerate}
\item[(a)] Define $n^- = |[0,\lambda_L] \cap X|$ and $n^+ = |[\lambda_R,1] \cap X|$. Then
$$
L_1(x)
\leq
\left\{
\begin{array}{ll}
\lg (n/n^+) & \mbox{if $x \geq \lambda_R$} \\
\lg (n/n^-) & \mbox{if $x \leq \lambda_L$}
\end{array}
\right.
$$
\item[(b)] Let $r(x)$ be the number of points between $x$ and the boundary point $\lambda$, counting $x$ itself. That is, $r(x) = |[x,\lambda) \cap X|$ if $x < \lambda$, or $|(\lambda, x] \cap X|$ if $x > \lambda$. Then $L_2(x) \leq \lg (n/r(x))$.
\end{enumerate}
\label{lemma:oned-monotonic-L}
\end{lemma}
\begin{proof}
For (a), take any $x \geq \lambda_R$ (the other case is similar). The interval $B = [\lambda,1]$ lies in $\B_\ell(x)$ for $\ell = \lceil \lg (n/n^+) - 1 \rceil$ and has $\eta_X(B) \geq \gamma$. Furthermore, any $B' \subset B$ also has $\eta_X(B') \geq \gamma$. 

For (b), take $x \geq \lambda_R$ and $\ell \geq \lg (n/r(x))$. Any $B \in \B_\ell(x)$ contains $< n/2^\ell \leq r(x)$ points and thus cannot possibly extend to the other side of the boundary. It follows that every $B \in \B_{\geq \ell}(x)$ has $\eta_X(B) \geq 0$. Moreover, any interval $B'$ that does extend to the other side of the boundary contains $B' \in \B_{\ell}(x)$ that is entirely on the same side as $x$.
\end{proof}

We can now bound the size of the query region at each level and find that it shrinks exponentially with $\ell$.
\begin{lemma}
For any $\ell \geq 0$, let $\Delta_\ell$ denote the focused querying region at level $\ell$, as defined in (\ref{eq:sampling-region}). Then $|\Delta_\ell| \leq 4n/2^\ell$.
\label{lemma:oned-monotonic-query-region}
\end{lemma}

\begin{proof}
We have
\begin{align*}
\Delta_\ell 
&= \bigcup_{x \in X: L_2(x) \geq \ell} \bigcup_{B \in \B_{\ell}(x)} (B \cap X) \\
&\subset \bigcup_{x \in X: r(x) \leq n/2^\ell} \bigcup_{B \ni x: |B \cap X| < n/2^\ell} (B \cap X) 
.
\end{align*}
This includes at most $n/2^{\ell-1}$ points from $X$ on either side of $\lambda$. \end{proof}


We are now ready for the proof of Theorem~\ref{thm:oned-monotonic}.

Setting $k$ to $O((1/\gamma^2) \ln (n/\delta))$ satisfies the requirements of Theorem~\ref{thm:label-complexity}. Here we are using the fact that although $\B$ is infinite, we need only consider $O(n^2)$ distinct intervals since $|X| = n$.

First observe that for any $\ell_1 \leq \ell_2$, we have from Lemma~\ref{lemma:oned-monotonic-query-region} that
$$ \sum_{\ell=\ell_1+1}^{\ell_2} |\Delta_\ell| \tau_\ell 
\ \leq \ \sum_{\ell=\ell_1+1}^{\ell_2} \frac{4n}{2^\ell} \cdot \frac{2^{\ell+2} k}{n} 
\ = \ 16k(\ell_2-\ell_1).$$
Now, let's define
$$ \ell_1 = \left\lfloor \lg \frac{m}{32k} \right\rfloor, 
\ \ \ \ell_2 = \min \left( \ell_1 + \frac{m}{128k}, \ \ \lg \frac{n}{2k} \right).$$
Then for any $x \in [0, \lambda_L] \cup [\lambda_R,1]$, we have
$$ \ell_1 = \left\lfloor \lg \frac{m}{32k} \right\rfloor 
\ \geq \ 
\lg \frac{m}{64k}
\ \geq \ 
\lg \frac{n}{\min(n^+,n^-)}
\ \geq \ 
L_1(x)$$
and, if $\min(r^+, r^-) \geq 2k$,
\begin{align*}
\ell_2 
\ = \ \ell_1 + \frac{m}{128k} 
&\geq \lg \frac{m}{64k} + \lg \frac{\min(n^+,n^-)}{\min(r^+,r^-)} \\
&\geq \ \lg \frac{m}{64k} + \lg \frac{64kn/m}{\min(r^+,r^-)} 
\ = \ \lg \frac{n}{\min(r^+,r^-)}
\ \geq \  
L_2(x).
\end{align*}
We get the algorithmic guarantee by applying Theorem~\ref{thm:label-complexity}. 

\subsection{One-dimensional data with Massart noise}
\label{sec:oned-massart}

We now turn to another one-dimensional setting. Once again, $X$ consists of $n$ arbitrarily-placed points in $\X = [0,1]$. This time, however, they are labeled according to a conditional probability function $\eta: \X \to [-1,1]$ that satisfies the Massart noise condition:
\begin{itemize}
\item There are $p$ disjoint open intervals $I_1, \ldots, I_p$, such that $\X$ is (the closure of) their union, and
\item for each $j$, either $\eta(x) > \gamma$ for all $x \in I_j$ or $\eta(x) < -\gamma$ for all $x \in I_j$. In the first case, we write $s(I_j) = +1$ and in the second case, $s(I_j) = -1$.
\end{itemize}
Here $0 < \gamma < 1$ is some constant. See Figure~\ref{fig:oned}(b) for an illustrative example. For concreteness, the intervals $I_j$ can be written in the form $(\lambda_{j-1}, \lambda_j)$, where
$0 = \lambda_0 < \lambda_1 < \cdots < \lambda_{p-1} < \lambda_p = 1 .$
Here $\lambda_1, \ldots, \lambda_{p-1}$ are the \emph{boundary points} between intervals.


We will take $\B$ to consist of all open intervals of $[0,1]$, with $\B(x)$ denoting intervals that contain point $x$. 

We begin with bounds on the $L_1$ and $L_2$ levels for each point.
\begin{lemma}
Pick any $x \in \X$; suppose $x \in I_j$.
\begin{enumerate}
\item[(a)] Let $n_j = |X \cap I_j|$. Then $L_1(x) \leq \lg (n/n_j)$.
\item[(b)] Let $r(x)$ be the minimum number of data points that lie between $x$ and a boundary point, counting $x$ as well; this is either the number of points in the left-interval $(\lambda_{j-1},x]$ (if $j > 1$) or the right-interval $[x, \lambda_j)$ (if $j < p$), whichever is smaller. Then $L_2(x) \leq \lg (n/r(x))$.
\end{enumerate}
\label{lemma:oned-massart-L}
\end{lemma}
\begin{proof}
For (a), notice first that $I_j \in \B_{\ell}(x)$ for $\ell = \lceil (\lg n/n_j) - 1 \rceil$. Moreover, $s(I_j) \cdot \eta_X(I_j) > \gamma$. Thus $I_j$ belongs to $\B_\ell(x)$ and is strongly biased towards the correct label. This strong bias also holds for any subset of $I_j$. 

For (b), consider any $\ell \geq \lg (n/r(x))$. Any $B \in \B(x)$ with $\mbox{sign}(\eta_X(B)) \neq s(I_j)$ must contain either the entire left-interval $(\lambda_{j-1},x]$ or the entire right-interval $[x, \lambda_j)$, and thus has at least $r(x)$ points, which means that it is too large to be in $\B_\ell$. Thus all intervals $B \in \B_{\geq \ell}(x)$ have $s(I_j) \cdot \eta_X(B) > 0$. Also, for any interval $B \in \B_{<\ell}(x)$ there is some $B' \in \B_{\ell}(x)$ that is strictly contained within it.
\end{proof}


With $L_1(x)$ and $L_2(x)$ under control, it is easy to bound the size of the focused query region $\Delta_\ell$ at each level.
\begin{lemma}
For any $\ell \geq 0$, let $\Delta_\ell$ denote the focused querying region at level $\ell$, as defined in (\ref{eq:sampling-region}). Then $|\Delta_\ell| \leq (p-1)n/2^{\ell-2}$.
\label{lemma:oned-massart-query-region}
\end{lemma}

\begin{proof}
We have
\begin{align*}
\Delta_\ell 
&= \bigcup_{x \in X: L_2(x) \geq \ell} \bigcup_{B \in \B_{\ell}(x)} (B \cap X) \\
&\subset \bigcup_{x \in X: r(x) \leq n/2^\ell} \bigcup_{B \ni x: |B \cap X| < n/2^\ell} (B \cap X) 
.
\end{align*}
This includes at most $n/2^{\ell-1}$ points from $X$ on either side of each boundary point $\lambda_j$.
\end{proof}

Notice that $|\Delta_\ell|$ shrinks exponentially with $\ell$. With $L_1$, $L_2$, and $|\Delta_\ell|$ values in place, Theorem~\ref{thm:label-complexity} can be applied directly to give the following label complexity bound.

\begin{thm}
Pick any $0 < \epsilon, \delta < 1$. Suppose we run the algorithm of Figure~\ref{alg:main} with $k = O((1/\gamma^2) \ln (n/\delta))$. With probability at least $1-\delta$, after making
$$ O \left( \frac{p-1}{\gamma^2} \ln \frac{p-1}{\epsilon} \ln \frac{n}{\delta} \right) $$
queries, the algorithm will assign the correct label $g^*(x)$ to at least $1-\epsilon$ fraction of $X$, except possibly the $2k$ points of either side of each boundary point.
\label{thm:oned-massart}
\end{thm}

\begin{proof}
Setting $k$ to $O((1/\gamma^2) \ln (n/\delta))$ satisfies the requirements of Theorem~\ref{thm:label-complexity}. Here we are using the fact that although $\B$ is infinite, we need only consider $O(n^2)$ distinct intervals since $|X| = n$.

Next, using Lemma~\ref{lemma:oned-massart-query-region}, we have that for any integers $0 \leq \ell_1 < \ell_2$,
$$ \sum_{\ell = \ell_1+1}^{\ell_2} |\Delta_\ell| \tau_\ell
\ \leq \ 
\sum_{\ell = \ell_1+1}^{\ell_2} \frac{(p-1) n}{2^{\ell-2}} \cdot \frac{2^{\ell+2}k}{n}
\ = \ 
16(p-1)k (\ell_2 - \ell_1).
$$
We can then apply Theorem~\ref{thm:label-complexity} to conclude that $m$ query points are enough to correctly classify all $x \in X$ with $L_1(x) \leq \ell_1$ and $L_2(x) \leq \ell_2$, for
\begin{align*}
\ell_1
&= 
\left\lfloor \lg \frac{m}{32k} \right\rfloor \\
\ell_2
&=
\min \left( \ell_1 + \frac{m}{128 k(p-1)}, \ \lg \frac{n}{2k} \right)
\end{align*}
Using Lemma~\ref{lemma:oned-massart-L}, we have that in every target interval $I_j$ with $n_j/n = \Omega(k/m)$, all but $n \cdot 2^{-\ell_2-1}$ points will be correctly classified. For large enough $m$, this means that the fraction of misclassified or unclassified points in $X$ will be at most $\epsilon$ after $O(k(p-1) \log ((p-1)/\epsilon))$ queries, apart from the $2k$ points nearest the boundaries, which will remain unclassified.
\end{proof}

The $2k$ points nearest each boundary cannot be labeled by our algorithm with any certainty because they lie in intervals with a strongly positive bias as well as in intervals with a strongly negative bias. This qualification would be removed if were allowed to make multiple queries to each point, because in that case we would include $O(k)$ copies of each point, as explained earlier.

\section{Analysis Details: distributional setting}

In the distributional setting, $X$ is drawn i.i.d.\ from a distribution $\mu$ on a topological space $\X$. We have a collection of regions $\B$.

\subsection{Sampling level and probability mass}

We start by showing how the fidelity of $\B$ (Definition~\ref{defn:conformity}) can be established.
\begin{lemma}
With probability at least $1-\delta$,
\begin{enumerate}
\item[(a)] For all $B \in \B$ with $n \mu(B) \geq 12 \ln (2|\B|/\delta)$, we have
$$ \frac{\mu(B)}{2} \leq \frac{|X_B|}{n} \leq 2 \mu(B) .$$
\item[(b)] For all $B \in \B$ with $n \mu(B) < 12 \ln (2|\B|/\delta)$, we have $|X_B| < 24 \ln (2 |\B|/\delta)$.
\end{enumerate}
\label{lemma:ball-size-bounds}
\end{lemma}
\begin{proof}
It is an immediate consequence of the multiplicative Chernoff bound that with probability at least $1-\delta$, for all $B \in \B$,
$$ \frac{|X_B|}{n} = \mu(B) \left(1 \pm \sqrt{\frac{3}{n \mu(B)} \ln \frac{2|\B|}{\delta}} \right) .$$
\end{proof}

In particular, taking $k > 24 \ln (2 |\B|/\delta)$ ensures that condition (a) hold will hold for all $B \in \B$ with $|X_B| \geq k$.

Next, we will see that balls of probability mass $p$ belong to level $\ell \approx \lg (1/p)$.
\begin{lemma}
For any level $\ell \geq 0$ and any $B \in \B_{\ell}$ with $|X_B| \geq k$,
$$ \left\lceil \lg \frac{1}{\mu(B)} \right\rceil -2 \leq \ell \leq  \left\lceil \lg \frac{1}{\mu(B)} \right\rceil.$$
\label{lemma:mass-vs-level}
\end{lemma}
\begin{proof}
Recall the definition of level $\ell$:
$$ B \in \B_\ell 
\ \Longleftrightarrow \ \frac{n}{2^{\ell+1}} \leq |X_B| < \frac{n}{2^\ell} 
\ \Longleftrightarrow \ 2^\ell < \frac{n}{|X_B|} \leq 2^{\ell + 1}.$$
By Lemma~\ref{lemma:ball-size-bounds}, 
$$ \frac{1}{2 \mu(B)} \leq \frac{n}{|X_B|} \leq \frac{2}{\mu(B)}.$$
Thus $2^{\ell} < 2/\mu(B)$ and $1/(2 \mu(B)) \leq 2^{\ell+1}$. These translate into the stated bounds on $\ell$.
\end{proof}

\subsection{Bounding $L_2$ using probabilistic distance: Proof of Lemma~\ref{lemma:L2-bound}}

Let $p = \dist(x, \X^{-s(x)}) \geq 2k/n$ and let $\ell = \lceil \lg (1/p) \rceil + 1 \leq \lg (2n/k)$. 

For any $B \in \B_{\geq \ell}(x)$, we have $\mu(B) \leq 2|X_B|/n < 2/2^\ell \leq p$, using fidelity (Definition~\ref{defn:conformity}), the definition of sampling levels, and the definition of $\ell$, in that order. It follows that $B$ does not intersect $\X^{-s(x)}$, whereupon $s(x) \cdot \eta_X(B) \geq 0$.

Next, pick any $B \in \B_{\leq \ell}(x)$ with $s(x) \cdot \eta_X(B) < 0$; thus $B \in \B_{< \ell}(x)$. By the richness of $\B$, there exists $B' \in \B_\ell(x)$ with $X_{B'} \subset X_B$; and so $s(x) \cdot \eta_X(B') \geq 0$. 
% Thus $B$ must intersect $\X^{-s(x)}$ and has probability mass $\geq p$. We will show that there exists $B' \in \B_{\leq \ell}(x)$ such that $B' \subset B$ and $B'$ does not intersect $\X^{-s(x)}$; whereupon $s(x) \cdot \eta_X(B') \geq 0$. Indeed, take $B' \in \B(x)$ to be a subset of $B$ of $\mu$-mass $p-\epsilon$ for some very small $\epsilon$; we can do this because of the absolute continuity of $\mu$. Then $B'$ does not touch $\X^{-s(x)}$ and by Lemma~\ref{lemma:mass-vs-level}, for small enough $\epsilon$, it lies at level $\leq \ell$.

\subsection{Uncertainty region in the distributional setting: Proof of Lemma~\ref{lemma:delta-continuous}}

Recall from (\ref{eq:sampling-region}) that
$$ \Delta_\ell = \bigcup_{x \in X: L_2(x) \geq \ell} \bigcup_{B \in \B_{\ell}(x)} X_B .$$
Consider any $z \in \Delta_\ell$. Then there exists $x \in \X$ with $L_2(x) \geq \ell$ and $B \in \B_\ell(x)$ such that $z \in B$. Now, $B \in \B_\ell(x)$ implies $|X_B|/n < 1/2^\ell$ and so (by the fidelity property) $\mu(B) < 2/2^\ell$. We will show that $x \in \partial_{4/2^\ell}$ and thus $z \in \partial_{4/2^\ell, 2/2^{\ell}}$.

There are two cases for $x$. If $\dist(x, \X^{-s(x)}) < 2k/n$ then we immediately have $x \in \partial_{4/2^\ell}$ since $4/2^\ell \geq 2k/n$. 

On the other hand, if $\dist(x, \X^{-s(x)}) \geq 2k/n$ then, since $L_2(x) \geq \ell$, we can apply Lemma~\ref{lemma:L2-bound} to get 
$$ \ell 
\ \leq \ \left\lceil \lg \frac{1}{\dist(x,\X^{-s(x)})} \right\rceil + 1 
\ \leq \ 
\lg \frac{1}{\dist(x,\X^{-s(x)})} + 2 .
$$
Thus $\dist(x, \X^{-s(x)}) \leq 1/2^{\ell-2}$ and $x \in \partial_{4/2^\ell}$.

\subsection{A generic label complexity bound in the distributional setting}

We will need to relate the size of the query region $\Delta_\ell$ to the probability mass of the corresponding second-order boundary. For this, we provide an analog of Lemma~\ref{lemma:level-distribution} for the distributional setting.
\begin{lemma}
With probability at least $1-2(\lg (n/k))e^{-k/3}$, the following hold for all levels $0 \leq \ell \leq \lg (n/4k)$.
\begin{enumerate}
\item[(a)] $|\{x \in X: T_x \leq \tau_\ell\}| \leq 2 n \tau_\ell$.
\item[(b)] $|\{x \in \Delta_{\ell}: T_x \leq \tau_\ell\}| \leq 2 n \mu(\partial_{4/2^\ell, 2/2^\ell}) \tau_\ell$.
\end{enumerate}
\label{lemma:level-distribution-dist}
\end{lemma}
\begin{proof} 
Part (a) is as in Lemma~\ref{lemma:level-distribution}. 

For (b), pick any subset $S \subset \X$. If $X$ consists of $n$ independent draws from $\mu$, then $|\{x \in S: T_x \leq \tau_\ell\}|$ has a $\mbox{binomial}(n, \mu(S) \tau_\ell)$ distribution with expectation $n \mu(S) \tau_\ell$. The probability that it is more than twice its expected value is, by a multiplicative Chernoff bound, at most $e^{-n \mu(S) \tau_\ell/3}$. We will apply this to the various sets $S = \partial_{4/2^\ell, 2/2^\ell}$ and take a union bound over them. In each application, we will also see that $n \mu(S) \tau_\ell \geq k$.

Pick any $\ell \leq \lg (n/4k)$. If $\partial_{4/2^\ell, 2/2^\ell} = \emptyset$, then the statement in (b) is trivially true given Lemma~\ref{lemma:delta-continuous}. So assume this is not the case. Writing $p = 4/2^\ell$, we need to check that $\mu(\partial_{p, p/2}) \tau_\ell \geq k/n$, or equivalently, $\mu(\partial_{p, p/2}) \geq \max(1/2^{\ell+2}, k/n) = 1/2^{\ell+2}$. Now, $\partial_{p,p/2} \neq \emptyset \Longrightarrow \partial_p \neq \emptyset$. Pick any $x \in \partial_p$. We have by the richness of $\B$ (Definition~\ref{defn:conformity}) that there exists $B \in \B_\ell(x)$. By the definition of level, $p/8 \leq |X_B|/n < p/4$ and thus, by the fidelity property, $p/16 \leq \mu(B) < p/2$. It follows that $B \subset \partial_{p,p/2}$, whereupon and $\mu(\partial_{p,p/2}) \geq \mu(B) \geq p/16 = 1/2^{\ell+2}$.
%By absolute continuity of $\mu$, we can grow a ball around $x$ of probability mass arbitrarily close to $p/2$, so that this ball is contained within $\partial_{p,p/2}$. Thus $\mu(\partial_{p,p/2}) \geq p/2-\epsilon$ for any $\epsilon > 0$. The required conditions then follow from the value of $p$.
\end{proof}

\begin{thm}
Suppose that $k \geq (192/\gamma^2) \ln (4 |\B|/\delta)$ and that the active learning algorithm makes $0 < m \leq n$ queries. Then with probability at least $1-3\delta$, all points $x \in X$ with $L_1(x) \leq \ell_1$ and $L_2(x) \leq \ell_2$ will get Bayes-optimal labels $\yh(x) = g^*(x)$, where 
$$ \ell_1 = \left\lfloor \lg \frac{m}{32k} \right\rfloor $$
and $\ell_2$ is the largest integer $\leq \lg (n/4k)$ such that
$$ \sum_{\ell = \ell_1 + 1}^{\ell_2} 2^\ell \mu(\partial_{4/2^\ell, 2/2^\ell}) \ < \ \frac{m}{32k} .$$
\label{thm:label-complexity-dist}
\end{thm}

\begin{proof}
The proof is identical to that of Theorem~\ref{thm:label-complexity-0}; the only change is to use Lemma~\ref{lemma:level-distribution-dist}(b) in place of Lemma~\ref{lemma:level-distribution}(b).
\end{proof}


\subsection{Rates of convergence under distributional conditions}

We now give label complexity bounds under three assumptions. Let $\X_\gamma = \{x \in \X: |\eta(x)| \geq \gamma\}$; thus $X \cap \X_\gamma$ is the set of points whose assigned labels will be judged. We further divide this set by label: for $s \in \{-1,+1\}$, let $\X^s_\gamma = \{x \in \X: s \cdot \eta(x) \geq \gamma\}$. 

The first assumption is on the curvature of the decision regions. It allows us to bound $L_1(x)$.
\begin{enumerate}
\item[(A1)] There is an absolute constant $p_o > 0$ for which the following holds: for any $x \in \X_\gamma$, there exists $B \in \B(x)$ such that $B \subset \X^{s(x)}_\gamma$ and $\mu(B) \geq p_o$.
\end{enumerate}
\begin{lemma}
Under (A1), every $x \in \X_\gamma$ has $L_1(x) \leq \lceil \lg (1/p_o) \rceil$, provided $n > k/2p_o$.
\label{lemma:L1-bound}
\end{lemma} 
\begin{proof}
Pick any $x \in X_\gamma$; apply (A1) to get $B \in \B(x)$ for which $\mu(B) \geq p_o$ and $B \subset \X^{s(x)}_\gamma$. By Lemma~\ref{lemma:mass-vs-level}, $B \in \B_\ell$ for $\ell \leq \lceil \lg (1/p_o) \rceil$. Now, $s(x) \cdot \eta_X(B) \geq \gamma$; moreover, for any $B' \in \B(x)$ with $X_{B'} \subset X_B$ we have $X_{B'} \subset \X^{s(x)}_\gamma$ and thus $s(x) \cdot \eta_X(B') \geq \gamma$ as well.
\end{proof}

The second assumption is a variant of the Tsybakov margin condition and bounds $|\Delta_\ell|$.
\begin{enumerate}
\item[(A2)] There are absolute constants $C > 0$ and $0 < \sigma < 1$ for which the following holds: for any $p > 0$, we have $\mu(\partial_{p,p}) \leq C p^\sigma$.
\end{enumerate}
If, for instance, $\mu$ were the uniform distribution over $[0,1]^d$, we would expect $\sigma \sim 1/d$.

The third assumption bounds the fraction of points with large $L_2$ values. It asks, what fraction of the $p$-boundary touches $\X_\gamma$? We consider two options: under (A3), the fraction is zero for $p$ small enough; under (A3'), the fraction is $p^\xi$.
\begin{enumerate}
\item[(A3)] There is an absolute constant $p_1 > 0$ such that $\mu(\partial_p \cap \X_\gamma) = 0$ for any $p < p_1$.
\item[(A3')] There are constants $C' > 0$ and $0 < \xi < 1$ such that $\mu(\partial_{p} \cap \X_\gamma) \leq C' p^\xi$ for any $p > 0$.
\end{enumerate}


Under these assumptions, we get rates of convergence as follows.
\begin{thm}
Assume (A1), (A2) and either (A3) or (A3'). There is an absolute constant $C''$ for which the following holds. Pick $0 < \delta < 1$ and take $k = O(((d \log n) + \log (1/\delta))/\gamma^2)$. If the algorithm of Fig.~\ref{alg:main} makes $m$ queries, for
$$ \frac{128k}{p_o} \ \leq \ m \ \leq \ \frac{64C \cdot 8^\sigma}{1-\sigma} \, n^{1-\sigma} k^\sigma,$$
then with probability at least $1-\delta$:
\begin{itemize}
\item Under (A3), all of $X \cap \X_\gamma$ get Bayes-optimal labels for $m > (512C/(1-\sigma)) \cdot (1/p_1)^{1-\sigma} \cdot k$.
\item Under (A3'),  Bayes-optimal labels get assigned to all but $ C'' (k/m)^{\xi/(1-\sigma)} + (2/n) \log (4/\delta)$
%$$ C'' \left(\frac{k}{m}\right)^{\xi/(1-\sigma)} + \frac{2 \log (4/\delta)}{n} $$
fraction of $X \cap \X_\gamma$.
\end{itemize}
\label{thm:label-complexity-specific}
\end{thm}
This theorem yields the result of Section~\ref{sec:Massart}. Another example is in Section~\ref{sec:Tsybakov}, where bounds are given under smoothness and Tsybakov margin conditions.

\begin{proof}
In this case, $\B$ is infinite, but by standard VC-dimension arguments there are only $O(n^{d+1})$ balls with distinct sets $X_B$. This governs the setting of $k$.

First, define $\ell_1 = \lfloor \lg (m/32k) \rfloor$ and observe that by Lemma~\ref{lemma:L1-bound}, all points in $\X_\gamma$ have 
$$ L_1(x) 
\ \leq \ 
\left\lceil \lg \frac{1}{p_o} \right\rceil
\ \leq \ 
\lg \frac{2}{p_o} 
\ \leq \ 
\lg \frac{m}{64k}
\ \leq \ 
\ell_1 .
$$
Next, pick  
$$ \ell_2 = \left\lfloor \frac{1}{1-\sigma} \left( \lg \frac{m}{32k} + \lg \frac{1-\sigma}{C \cdot 4^{1+\sigma}} \right) \right\rfloor.$$
The upper bound on $m$ ensures that this is at most $\lg (n/2k)$. Then, using (A2),
$$
\sum_{\ell = \ell_1+1}^{\ell_2} 2^\ell \mu(\partial_{4/2^\ell, 2/2^\ell})
\ \leq \ 
C \sum_{\ell = \ell_1+1}^{\ell_2} 2^\ell \left( \frac{4}{2^\ell} \right)^\sigma 
\ \leq \ 
C \cdot 4^\sigma \cdot \frac{2}{2^{1-\sigma}-1} \cdot 2^{(1-\sigma)\ell_2}
\ \leq \ 
\frac{m}{32k} .
$$
Applying Theorem~\ref{thm:label-complexity-dist}, we then see that all points in $\X_\gamma$ with $L_2(x) \leq \ell_2$ will be correctly classified. Any remaining point $x \in \X_\gamma$ has $L_2(x) > \ell_2$ and thus (by Lemma~\ref{lemma:L2-bound})
$$ \left\lceil \lg \frac{1}{\dist(x, \X^{-s(x)})} \right\rceil + 1 > \ell_2 
\ \Longrightarrow \ 
\dist(x, \X^{-s(x)}) < \frac{4}{2^{\ell_2}} \leq  \left( \frac{512C}{1-\sigma} \right)^{1/(1-\sigma)}  \cdot \left( \frac{k}{m} \right)^{1/(1-\sigma)} . 
$$
Call this quantity $p$; thus any such $x$ lies in $\partial_p$.

Under (A3), $\partial_p \cap \X_\gamma$ has zero probability mass for $p < p_1$, that is, if 
$$ \left( \frac{512C}{1-\sigma} \right)^{1/(1-\sigma)}  \cdot \left( \frac{k}{m} \right)^{1/(1-\sigma)} < p_1
\ \Longleftrightarrow \ 
m > \frac{512C}{1-\sigma} \cdot \frac{1}{p_1^{1-\sigma}} \cdot k .
$$

Under (A3'), $\mu(\partial_p \cap \X_\gamma) \leq C' p^{\xi}$; we can then apply a Bernstein bound to assert that with probability at least $1-\delta$, 
$$ |X \cap (\partial_p \cap \X_\gamma)| \ \leq \ \frac{3}{2} C' n p^{\xi} + 2 \log \frac{1}{\delta},$$
from which the bound in the theorem follows by defining $C''$ appropriately.
\end{proof}

\subsection{Label complexity under curvature and Massart noise: Proof of Theorem~\ref{thm:massart-dist}}

Let (C1) denote the strong density condition and (C2) the Massart noise and boundary condition. Theorem~\ref{thm:massart-dist} follows immediately from the more general result of Theorem~\ref{thm:label-complexity-specific}, once we establish how (C1) and (C2) relate to assumptions (A1), (A2), and (A3').

\begin{lemma}
Conditions (C1) and (C2) yield assumption (A1), with $p_o = r_o^d \cdot c_o \cdot v_d$, where $v_d$ is the volume of the unit ball in $\R^d$.
\label{lemma:A1}
\end{lemma}

\begin{proof}
Pick any $x \in \X_\gamma^+$ (the negative case is similar), and let $r = \inf_{z \in \X^0} \|x - z\|$. If $r \geq r_o$, then $B = B(x,r_o)$ is entirely in $\X_\gamma^+$. Otherwise, the reach condition (C2) implies the existence of a ball $B \subset \X_\gamma^+$ that contains $x$ and has radius $r_o$. Either way, $\mu(B) \geq c_o \vol(B) = c_o v_d r_o^d$ by (C1). 
\end{proof}

\begin{lemma}
Conditions (C1) and (C2) yield assumptions (A2) and (A3') with $\sigma = \xi = 1/d$. 
\end{lemma}

\begin{proof}
Under (C1), any ball of probability mass $\leq p$ has volume $\leq p/c_o$ and radius $\leq (p/(c_o v_d))^{1/d}$. Let's call this latter quantity $r$. Thus, any point in $\partial_p$ lies within distance $2r$ of the boundary, while a point in $\partial_{p,p}$ lies within distance $4r$. 

To bound the volume of $\partial_{p,p}$, we can associate each point in this region with its nearest neighbor in $\X^0$; by condition (C1), this projection map is uniquely defined for $r < r_o/4$. The volume of the region is thus $O(r)$ and under (C1), has probability mass $O(r) = O(p^{1/d})$.
\end{proof}


\subsection{Label complexity under smoothness and Tsybakov noise}
\label{sec:Tsybakov}

Now we specialize Theorem~\ref{thm:label-complexity-specific} to another situation, where the data distribution satisfies Holder smoothness and Tsybakov margin conditions.

Recall that $\dist(x,S)$ denotes the probability-distance from $x$ and set $S$. We will overload notation so that for $x,x' \in \X$,
$$ \dist(x,x') = \dist(x, \{x'\}) = \inf\{\mu(B): B \in \B(x) \cap \B(x')\},$$
that is, the probability mass of the smallest ball containing both $x$ and $x'$. We will impose the following conditions on the data.
\begin{enumerate}
\item[(C1)] [Strong density condition] This was introduced in the earlier (Massart) example.
\item[(C2')] [Holder-smoothness of conditional probability function] There exist constants $L, \alpha$ such that
$$ |\eta(x) - \eta(x')| \ \leq \ L \cdot \dist(x,x')^\alpha$$
for all $x,x' \in \X$.
\item[(C3')] [Tsybakov margin condition] There exists constants $M, \beta$ such that
$$ \mu(\{x \in \X: |\eta(x)| \leq \tau\}) \leq M \tau^\beta$$
for all $\tau \in (0,1)$.
\item[(C4')] [Bounded curvature] The boundaries $\{x \in \X: \eta(x) = \gamma\}$ and $\{x \in \X: \eta(x) = -\gamma\}$ are $(d-1)$-dimensional Riemannian manifolds of reach $r_o > 0$.
\end{enumerate}

Lemma~\ref{lemma:A1} continues to hold, with condition (C4') doing the job of (C2). This yields assumption (A1). For the remaining assumptions, we first obtain a consequence of the Holder condition.

\begin{lemma}
Under (C2'), for any $p, q > 0$,
\begin{enumerate}
\item[(a)] $x \in \partial_p \implies |\eta(x)| \leq L p^\alpha$.
\item[(b)] $x \in \partial_{p,q} \implies |\eta(x)| \leq L(p^\alpha + q^\alpha)$.
\end{enumerate}
\label{lemma:holder-consequence}
\end{lemma}
\begin{proof}
Pick any $x \in \partial_p$. Let $s(x) = \mbox{sign}(\eta(x))$. By definition of the $p$-boundary set, for any $\epsilon > 0$, there exists $x' \in \X^{-s(x)}$ such that $\dist(x,x') < p+\epsilon$. By the Holder condition, $|\eta(x) - \eta(x')| < L(p+\epsilon)^\alpha$ and thus $|\eta(x)| < L(p+\epsilon)^\alpha$. Since this holds for any $\epsilon > 0$, we get part (a).

For (b), pick $x \in \partial_{p,q}$ and $\epsilon > 0$. Then there exists $x' \in \partial_p$ with $\dist(x,x') < q+\epsilon$. As before, we use the Holder condition to conclude that $|\eta(x)| < |\eta(x')| + L(q+\epsilon)^\alpha$ and then invoke (a).
\end{proof}

\begin{lemma}
Conditions (C2') and (C3') yield assumptions (A2) and (A3) with $C = (2L)^\beta M$, $\sigma = \alpha \beta$, and $p_1 = (\gamma/L)^{1/\alpha}$.
\label{lemma:boundary-constants} 
\end{lemma}

\begin{proof}
By Lemma~\ref{lemma:holder-consequence}, $\partial_{p,p} \subset \{x \in \X: |\eta(x)| \leq 2Lp^\alpha\}$; the probability mass of this set can be bounded by (C3').

Also by Lemma~\ref{lemma:holder-consequence}, $\partial_p$ is entirely contained in $\{x \in \X: |\eta(x)| \leq L p^\alpha\}$. For $p < p_1$, this does not intersect $\X_\gamma$.
\end{proof}

\begin{thm}
Assume conditions (C1) and (C2')--(C4'). There are constants $c_2, c_3$ for which the following holds. Pick $0 < \delta < 1$ and take $k = O(((d \log n) + \log (1/\delta))/\gamma^2)$. Suppose the algorithm of Fig.~\ref{alg:main} makes $m$ queries, where
$$ \frac{c_2k}{p_1^{1-\sigma}} \ \leq \ m \ \leq \ c_3 n^{1-\sigma} ,$$
with $p_1, \sigma$ as given in Lemma~\ref{lemma:boundary-constants}. Then with probability at least $1-\delta$, all of $X \cap \X_\gamma$ is assigned Bayes-optimal labels.
\label{thm:smoothness-dist}
\end{thm}

In this setting, regular $k$-nearest neighbor classification would also correctly classify all of $X \cap X_\gamma$ if given $O(k/p_1)$ random labeled points. The benefit of active learning is to thus reduce the label complexity to $O(k/p_1^{1-\alpha\beta})$.
\end{document}




\section{FORMATTING INSTRUCTIONS}

To prepare a supplementary pdf file, we ask the authors to use \texttt{aistats2024.sty} as a style file and to follow the same formatting instructions as in the main paper.
The only difference is that the supplementary material must be in a \emph{single-column} format.
You can use \texttt{supplement.tex} in our starter pack as a starting point, or append the supplementary content to the main paper and split the final PDF into two separate files.

Note that reviewers are under no obligation to examine your supplementary material.

\section{MISSING PROOFS}

The supplementary materials may contain detailed proofs of the results that are missing in the main paper.

\subsection{Proof of Lemma 3}

\textit{In this section, we present the detailed proof of Lemma 3 and then [ ... ]}

\section{ADDITIONAL EXPERIMENTS}

If you have additional experimental results, you may include them in the supplementary materials.

\subsection{The Effect of Regularization Parameter}

\textit{Our algorithm depends on the regularization parameter $\lambda$. Figure 1 below illustrates the effect of this parameter on the performance of our algorithm. As we can see, [ ... ]}

\vfill

\end{document}

%%% Local Variables:
%%% mode: latex
%%% TeX-master: "aistats"
%%% End:
